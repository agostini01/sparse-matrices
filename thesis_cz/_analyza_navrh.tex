\chapter{Analýza a návrh}

V této kapitole představíme některé známé systémy, které umí pracovat s řídkými maticemi. Popíšeme možnosti implementace a zvolené řešení.

Pokud potřebujeme rychle vynásobit dvě malé matice, můžeme použít obecný nástroj, se kterým se snadno pracuje. Čím je ale nástroj obecnější, tím spíš bude pomalejší. Některé nástroje jsou ale specificky optimalizované a i přes jejich obecnost mají dobrou výkonnost. Maximální efektivnosti dosáhneme, pokud co nejvíce využijeme hardware a náš program bude řešit konkrétní úkol. 

\section{Software pro práci s řídkými maticemi}

\subsection{SciPy.sparse}

SciPy je knihovna skriptovacího jazyka Python \cite{Python}. Obsahuje modul pro základní práci s řídkými maticemi, nazvaný sparse \cite{scipy}. Samotné výpočty nad řídkými maticemi jsou v této knihovně z důvodu rychlosti napsány v jazyce C++.

\subsection{Boost}

Pro práci s řídkými maticemi v jazyce C++ je možné použít knihovnu uBLAS\cite{ublas} ze sady knihoven Boost\cite{boost}. Knihovna uBLAS obsahuje algoritmy pro řešení základních ukolů z lineálrní algebry. Protože je kód napsaný pomocí C++ šablon, je vygenerovaný kód výkonný.

\subsection{ATLAS}

ATLAS \cite{atlas} je v překladu zkratkou pro automatické generování vylazeného kódu pro software řešící úlohy z lineární algebry. Je tak dosaženo pomocí jak obecných optimalizací kódu, tak optimalizací pro konkrétní architektury.

Podobných projektů je více, konkrétně pro násobení matice s vektorem je to například Sparsity \cite{sparsity}, nebo pro násobení matice s maticí je to například PHiPAC \cite{PHiPAC}\cite{bilmes96a}.

\subsection{Wolfram Mathematica}

Wolfram Mathematica \cite{mathematica} je mocný výpočetní software. Práce v tomto software je snadná, interaktivní a přitom dokáže efektivně rozdělovat práci nejen na více procesorových jader, ale taky využít i grafické karty. Nechybí tedy ani implementace operací pro řídké matice.

V této práci pomocí Wolfram Mathematicy vizualizujeme řídké matice z formátu \texttt{.mtx}, který bude popsán v \ref{MM}. Mathematica umí pracovat i s maticemi v souborech \texttt{.mtx.gz}, tedy komprimovanými soubory. Příkaz pro importování řídké matice a její vizualizaci je \texttt{Import["/var/tmp/matrix.mtx", "Graphics"]}, následné uložení matice lze provést například příkazem \\ \texttt{Export["/var/tmp/matrix.pdf", \%]} \cite{mathematicaMTX}.

\section{Možnosti řešení}

Jedna z možných variant pro porovnání všech zvolených formátů byla doimplementovat formát Quadtree, respektive KAT do knihovny SciPy.sparse.

Další možnost byla pokračovat s semestrální práci z předmětu Efektivní implementace algoritmů s tématem násobení matic ve formátu CSR.

\section{Zvolené řešení}

Protože se v této práci zabýváme pouze vlivem uložení řídké matice pro operaci násobení, rozhodli jsme se vytvořit program jen pro tuto operaci. Obecná knihovna pro práci s řídkými maticemi, jakým například SciPy.sparse je, nemůže využít některých vlastností matic při jejich násobení. Hlavně zacházení s násobenými maticemi jako s konstantami.

Navážeme tedy na semestrální práci. Vytvoříme více druhů matic a pro každou implementujeme operaci násobení s maticí a násobení s vektorem.
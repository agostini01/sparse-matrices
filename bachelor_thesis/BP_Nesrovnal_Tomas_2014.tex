% options:
% thesis=B bachelor's thesis
% thesis=M master's thesis
% czech thesis in Czech language
% english thesis in English language
% hidelinks remove colour boxes around hyperlinks

\documentclass[thesis=B,english]{FITthesis}[2012/10/20]

% \usepackage[utf8]{inputenc} % LaTeX source encoded as UTF-8
% \usepackage[latin2]{inputenc} % LaTeX source encoded as ISO-8859-2
% \usepackage[cp1250]{inputenc} % LaTeX source encoded as Windows-1250

\usepackage{graphicx} %graphics files inclusion
% \usepackage{subfig} %subfigures
% \usepackage{amsmath} %advanced maths
% \usepackage{amssymb} %additional math symbols

\usepackage{dirtree} %directory tree visualisation

% % list of acronyms
% \usepackage[acronym,nonumberlist,toc,numberedsection=autolabel]{glossaries}
% \iflanguage{czech}{\renewcommand*{\acronymname}{Seznam pou{\v z}it{\' y}ch zkratek}}{}
% \makeglossaries

% % % % % % % % % % % % % % % % % % % % % % % % % % % % % % 
% EDIT THIS
% % % % % % % % % % % % % % % % % % % % % % % % % % % % % % 

\department{Department of Theoretical Computer Science}
\title{Impact of sparse matrix storage format on efficiency of multiplication of sparse matrices}
\authorGN{Tom{\' a}{\v s}} %author's given name/names
\authorFN{Nesrovnal} %author's surname
\author{Tom{\' a}{\v s} Nesrovnal} %author's name without academic degrees
\authorWithDegrees{Tom{\' a}{\v s} Nesrovnal} %author's name with academic degrees
\supervisor{Ivan Šimeček}
\acknowledgements{THANKS}
\abstractEN{Summarize the contents and contribution of your work in a few sentences in English language.}
\abstractCS{V n{\v e}kolika v{\v e}t{\' a}ch shr{\v n}te obsah a p{\v r}{\' i}nos t{\' e}to pr{\' a}ce v {\v c}esk{\' e}m jazyce.}
\placeForDeclarationOfAuthenticity{Prague}
\keywordsCS{Replace with comma-separated list of keywords in Czech.}
\keywordsEN{Replace with comma-separated list of keywords in English.}
\declarationOfAuthenticityOption{1} %select as appropriate, according to the desired license (integer 1-6)


\begin{document}

% \newacronym{CVUT}{{\v C}VUT}{{\v C}esk{\' e} vysok{\' e} u{\v c}en{\' i} technick{\' e} v Praze}
% \newacronym{FIT}{FIT}{Fakulta informa{\v c}n{\' i}ch technologi{\' i}}

\setsecnumdepth{part}
\chapter{Introduction}

\section{Matrix multiplication}

Matrix multiplication is one of the basic  linear algebra operations.

\setsecnumdepth{all}
\chapter{State-of-the-art}

\chapter{Analysis and design}

\chapter{Sparse matrices}

\section{Sparsity}

A matrix is called sparse, when contains a lot of zero elements.

When working with sparse matrices, we can take in advantage its zero elements. In case of sparse matrix multiplication, we can use sparsity in two ways. First, a sparse matrix storage format doesn't have to store zero elements. Second, we can skip multiplying by zero elements.

The disadvantage of sparse matrix storage formats is we cannot directly get a single element by its x and y coordinate.

\section{Coordinate format}

\section{Compressed sparse rows}

\section{Block Sparse Rows}

\section{Quadtree}



\chapter{Realisation}

\setsecnumdepth{part}
\chapter{Conclusion}


\bibliographystyle{iso690}
\bibliography{mybibliographyfile}

\setsecnumdepth{all}
\appendix

\chapter{Acronyms}
% \printglossaries
\begin{description}
	\item[SpMMM] Sparse matrix-matrix multiplication
	\item[SpMVM] Sparse matrix-vector multiplication
\end{description}


\chapter{Contents of enclosed CD}

%change appropriately

\begin{figure}
	\dirtree{%
		.1 readme.txt\DTcomment{the file with CD contents description}.
		.1 exe\DTcomment{the directory with executables}.
		.1 src\DTcomment{the directory of source codes}.
		.2 wbdcm\DTcomment{implementation sources}.
		.2 thesis\DTcomment{the directory of \LaTeX{} source codes of the thesis}.
		.1 text\DTcomment{the thesis text directory}.
		.2 thesis.pdf\DTcomment{the thesis text in PDF format}.
		.2 thesis.ps\DTcomment{the thesis text in PS format}.
	}
\end{figure}

\end{document}
